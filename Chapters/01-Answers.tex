\chapter{Answers}
\label{cp:answers}
\section{Question 1}

\begin{importantbox}
    You should review and understand the concepts of wind tunnels.
\end{importantbox}

Engineers use wind tunnels to simulate and study the effects of airflow on scaled-down models of large objects—e.g., airfoils, wings, cars, objects, etc.—in a controlled environment. Engineers can use the data collected from wind tunnel testing to analyze aerodynamic performance—e.g., lift, drag, pressure, etc.—and an object's or system's dynamics. In addition to analysis, wind tunnel testing facilitates the design and optimization of aerodynamic components or systems.

\section{Question 2}

\begin{importantbox}
    You should understand the basic concepts of wind turbines.
\end{importantbox}

Wind turbines have been used for hundreds of years to convert wind energy to mechanical energy. Most wind turbines consist of three blades, each shaped like an airfoil. As air passes over the blades, they generate lift perpendicular to the blade, causing a torque on the generator.

The speed of the blades is limited to reduce mechanical noise and improve structural stability, but this slower rate isn't fast enough to generate substantial energy from the generator. To account for this slower rotational speed, a gear box with a high gear ratio is used to convert the slow rotation of the blade into a high rotation of the generator.

Wind turbines are optimized to generate the most mechanical energy when they are oriented normal to the wind direction. Of course, the wind direction changes. To account for these changes, a velocity sensor on top of the turbine nacelle tracks the wind speed and direction and sends signals to a motor that rotates the entire turbine. Additionally, each of the blades can rotate to be oriented at the optimal angle of attack.